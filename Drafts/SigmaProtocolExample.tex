\documentclass[11pt]{article}
\usepackage{amsmath}
\usepackage{breqn}

\begin{document}

\title{$\Sigma$-Protocol Example}
\author{David Mis}

\maketitle

The purpose of this document is to demonstrate my understanding of proofs-of-knowledge based on $\Sigma$-protocols. I'll do this by describing Protocol 1 of Camenisch et al, which is treated as a black-box and the details are left to the reader. Some placeholder names have been changed for clarity since we are considering Camenisch's scheme out of context.

Let $n$ be an RSA modulus, and $QR_n$ be the group of quadratic residues modulo $n$. Let $g, h$ be quadratic residues in $QR_n$.

The protocol is executed between a prover $P$ and a verifier $V$. Both $P$ and $V$ know $n, g$ and $h$.

The protocol is set up in the following manner: $P$ secretly chooses four random integers $a, b, c,$ and $d$. $P$ then sets $X_1 = g^{a}h^{b}$ and $X_2 = g^ch^{d}$. $P$ sends $X_1$ and $X_2$ to $V$, and wants to prove that they are formed correctly -- that is, $X_1$ and $X_2$ are both the product of a power of $g$ and a power of $h$. Using Camenisch's notation, $P$ and $V$ will engage in the $\Sigma$-protocol:

\begin{equation}
	PK\{(\alpha,\beta,\gamma,\delta):X_{1} = (g)^\alpha (h)^\beta \wedge X_2 = (g)^\gamma(h)^\delta\}.
\end{equation}

The above notation denotes a protocol where $P$ proves to $V$ that it knows values for $\alpha, \beta, \gamma$ and $\delta$ subject to the two equations on the right. However, $U$ can not reveal any information about $\alpha, \beta, \gamma$ or $\delta$ during the proof.

Camenisch actually proposes the following protocol
\begin{equation}
	PK\{(\alpha,\beta,\gamma,\delta):X_{1}^{2} = (g^2)^\alpha (h^2)^\beta \wedge X_2^{2} = (g^2)^\gamma(h^2)^\delta\}.
\end{equation}
It is clear that the two protocols are equivalent. It may be possible to execute the latter protocol faster, but I will focus on the former since it is slightly simpler to describe.

The $\Sigma$-protocol is executed over three rounds -- commitment, challenge and response. During the commitment round, $P$ chooses four new random integers $R_1, R_2, R_3, R_4$. $P$ then commits to these random values by setting

\begin{eqnarray}
	t_1=g^{R_1}, \\
	t_2=h^{R_2}, \\
	t_3=g^{R_3}, \\
	t_4=h^{R_4}. 
\end{eqnarray}

$P$ sends $t_1, t_2, t_3$ and $t_4$ to $V$. 

Now the challenge round begins, which is quite simple: $V$ chooses four random challenges $k_1, k_2, k_3$ and $k_4$, then sends them all to $P$.

Finally, the response round begins. $P$ sets the following values:

\begin{eqnarray}
	s_1 = R_1 + ak_1 + ak_2, \\
	s_2 = R_2 + bk_1 + bk_2, \\
	s_3 = R_3 + ck_3 + ck_4, \\
	s_4 = R_4 + dk_3 + dk_4.
\end{eqnarray}

$P$ sends all values to $V$, who can now check whether $g^{s_1}h^{s_2} \stackrel{?}{=} t_1(X_1^{k_1})t_2(X_1^{k_2})$ and $g^{s_3}h^{s_4} \stackrel{?}{=} t_3(X_2^{k_3})t_4(X_2^{k_4})$. $V$ accepts the proof if both equations hold and rejects otherwise. 

If $P$ executed the protocol correctly, then $V$ will accept the proof since:

\begin{dmath}
	(g^{s_1})(h^{s_2}) = (g^{R_1 + ak_1 + ak_2})(h^{R_2 + bk_1 + bk_2}) \\
	= (g^{R_1}) (g^{ak_1}) (g^{ak_2}) (h^{R_2}) (h^{bk_1}) (h^{bk_2}) \\
	= t_1 (g^{ak_1}) (h^{bk_1}) t_2 (g^{ak_2}) (h^{bk_2}) \\
	=  t_1 (g^ah^b)^{k_1} t_2 (g^ah^b)^{k_2} \\
	=  t_1 (X_1^{k_1}) t_2 (X_1^{k_2})
\end{dmath}

A similar series of substitutions holds for the other equation in question. 

I will now argue that if $V$ accepts the proof, then $U$ must know $a$ and $b$. If $V$ accepts the proof, then $P$ has produced an $s_1$ and $s_2$ that satisfy  $g^{s_1}h^{s_2} = t_1(X_1^{k_1})t_2(X_1^{k_2})$. From this, both $P$ and $V$ know $s_1 = R_1 + ak_1 + ak_2$. Since $P$ knows $s_1$, $R_1$ and $k_1$ by construction, $P$ must also know $a$ (even if $P$ forgot $a$, it could derive it from this equation). A similar argument shows $U$ must know $b$.
	
Finally, I claim the above protocol does not reveal any information about $a$ to $V$. As stated before, $V$ knows $s_1 = R_1 + ak_1 + ak_2$, but it is not able to recover $a$ since it does not know $R_1$. Furthermore, $V$ is not able to easily recover $R_1$ from $t_1$ due to the discrete logarithm problem. Similar arguments hold for the remaining values $b$, $c$ and $d$.
\end{document}


